%% Mathematics documenting solution to perspective ray tracing, part 1.
%%
%% To generate the PNG web image from this document, run the following
%% commands in order:
%%   latex persp_math1.tex
%%   dvips persp_math1.dvi -o persp_math1.ps
%%   convert -density 161x161 -trim persp_math1.ps persp_math1.png
%%   rm persp_math1.aux persp_math1.log persp_math1.dvi \
%%     persp_math1.ps
%%
%% Optionally, you might want to make the image have an opaque white
%% background so that the image does cause problems for IE6 users.
%% You should also keep the alpha channel in the PNG rather than
%% removing it, since for some odd reason, some versions of
%% semi-modern software don't cope well with rendering PNG images with
%% no alpha channel.
%%
%% blahtexml <http://gva.noekeon.org/blahtexml> can be used to
%% generate MathML from this LaTeX code.  To generate the MathML
%% fragments from this document, start blahtexml with the `--mathml'
%% option and run each portion of mathematics between the `\[' and
%% `\]' that is marked with ``New blahtexml run'' through blahtexml.
%% This will give you back a series of MathML fragments.
%%
%% After that, you will have to merge the fragments into
%% `persp_math1.xhtml'.
\documentclass{article}
\usepackage{amsmath}
\pagestyle{empty}
\begin{document}

\[ %% New blahtexml run.
\begin{pmatrix}
x^2 + y^2 = r^2 \\
y = mx + b
\end{pmatrix}
\]

\[ %% New blahtexml run.
\begin{aligned}
m &= -f/p \\
b &= r + d \\
x &= \frac{y - b}{m} \\
x &= -\frac{p}{f} (y - (r + d)) \\
x &= \frac{p}{f} (-y + (r + d))
\end{aligned}
\]

\[ %% New blahtexml run.
\begin{aligned}
\left(\frac{p}{f} (-y + (r + d))\right)^2 + y^2 &= r^2 \\
\left(\frac{p}{f}\right)^2 (-y + (r + d))^2 + y^2 &= r^2 \\
\left(\frac{p}{f}\right)^2 (y^2 - 2(r + d)y + (r + d)^2) + y^2 &= r^2 \\
\left(1 + \left(\frac{p}{f}\right)^2\right)y^2 -
  2 \left(\frac{p}{f}\right)^2 (r + d)y +
  \left(\frac{p}{f}\right)^2 (r + d)^2 - r^2 &= 0
\end{aligned}
\]

\[ %% New blahtexml run.
\begin{aligned}
a &= 1 + \left(\frac{p}{f}\right)^2 \\
b &= -2\left(\frac{p}{f}\right)^2 (r + d) \\
c &= \left(\frac{p}{f}\right)^2 (r + d)^2 - r^2 \\
y &= \frac{-b \pm \sqrt{b^2 - 4ac}}{2a} \\
y &= \frac{-b + \sqrt{b^2 - 4ac}}{2a}
\end{aligned}
\]

\end{document}
