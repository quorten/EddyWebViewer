%% Mathematics documenting solution to perspective ray tracing, part 2.
%%
%% To generate the PNG web image from this document, run the following
%% commands in order:
%%   latex persp_math2.tex
%%   dvips persp_math2.dvi -o persp_math2.ps
%%   convert -density 161x161 -trim persp_math2.ps persp_math2.png
%%   rm persp_math2.aux persp_math2.log persp_math2.dvi \
%%     persp_math2.ps
%%
%% Optionally, you might want to make the image have an opaque white
%% background so that the image does cause problems for IE6 users.
%% You should also keep the alpha channel in the PNG rather than
%% removing it, since for some odd reason, some versions of
%% semi-modern software don't cope well with rendering PNG images with
%% no alpha channel.
%%
%% blahtexml <http://gva.noekeon.org/blahtexml> can be used to
%% generate MathML from this LaTeX code.  To generate the MathML
%% fragments from this document, start blahtexml with the `--mathml'
%% option and run each portion of mathematics between the `\[' and
%% `\]' that is marked with ``New blahtexml run'' through blahtexml.
%% This will give you back a series of MathML fragments.
%%
%% After that, you will have to merge the fragments into
%% `persp_math2.xhtml'.
\documentclass{article}
\usepackage{amsmath}
\pagestyle{empty}
\begin{document}

\[ %% New blahtexml run.
\begin{pmatrix}
x^2 + y^2 + z^2 = r^2 \\
x = x_p / f \cdot (-z + (r + d)) \\
y = y_p / f \cdot (-z + (r + d))
\end{pmatrix}
\]

\[ %% New blahtexml run.
\begin{aligned}
\left(\frac{x_p}{f} (-z + (r + d)) \right)^2 +
  \left(\frac{y_p}{f} (-z + (r + d)) \right)^2 + z^2 &= r^2 \\
\frac{x_p^2 + y_p^2}{f^2} (-z + (r + d))^2 + z^2 &= r^2 \\
\frac{x_p^2 + y_p^2}{f^2} (z^2 - 2(r + d)z + (r + d)^2) + z^2 &= r^2 \\
\text{let} \ w = \frac{x_p^2 + y_p^2}{f^2} \\
(1 + w) z^2 - 2w(r + d)z + w(r + d)^2 - r^2 &= 0
\end{aligned}
\]

\[ %% New blahtexml run.
\begin{aligned}
a &= 1 + w \\
b &= -2w(r + d) \\
c &= w(r + d)^2 - r^2 \\
z &= \frac{-b \pm \sqrt{b^2 - 4ac}}{2a} \\
z &= \frac{-b + \sqrt{b^2 - 4ac}}{2a}
\end{aligned}
\]

\end{document}
