%% Mathematics documenting solution to perspective ray tracing.
%%
%% To generate the PNG web image from this document, run the following
%% commands in order:
%%   latex perspective_math.tex
%%   dvips perspective_math.dvi -o perspective_math.ps
%%   convert -density 161x161 -trim perspective_math.ps perspective_math.png
%%   rm perspective_math.aux perspective_math.log perspective_math.dvi \
%%     perspective_math.ps
%%
%% Optionally, you might want to make the image have an opaque white
%% background so that the image does cause problems for IE6 users.
%% You should also keep the alpha channel in the PNG rather than
%% removing it, since for some odd reason, some versions of
%% semi-modern software don't cope well with rendering PNG images with
%% no alpha channel.
%%
%% blahtexml <http://gva.noekeon.org/blahtexml> can be used to
%% generate MathML from this LaTeX code.  To generate the MathML
%% fragments from this document, start blahtexml with the `--mathml'
%% option and run each portion of mathematics between the `\[' and
%% `\]' that is marked with ``New blahtexml run'' through blahtexml.
%% This will give you back a series of MathML fragments.
%%
%% After that, you will have to merge the fragments into
%% `perspective_math.xhtml'.
\documentclass{article}
\usepackage{amsmath}
\pagestyle{empty}
\begin{document}

\[ %% New blahtexml run.
\begin{pmatrix}
x^2 + y^2 = r^2 \\
y = mx + b
\end{pmatrix}
\]

\[ %% New blahtexml run.
\begin{aligned}
m &= f/p \\
b &= r + d
\end{aligned}
\]

\[ %% New blahtexml run.
\begin{aligned}
x^2 + \left(\frac{f}{p}x + (r + d)\right)^2 &= r^2 \\
x^2 + \left(\frac{f}{p}\right)^2 x^2 + 2\left(\frac{f}{p}\right)(r + d)x +
  (r + d)^2 &= r^2 \\
\left(1 + \left(\frac{f}{p}\right)^2\right)x^2 +
  2\left(\frac{f}{p}\right)(r + d)x +
  (r^2 - r^2 + 2rd + d^2) &= 0
\end{aligned}
\]

\[ %% New blahtexml run.
\begin{aligned}
a &= 1 + \left(\frac{f}{p}\right)^2 \\
b &= 2\left(\frac{f}{p}\right)(r + d) \\
c &= 2rd + d^2 \\
x &= \frac{-b \pm \sqrt{b^2 - 4ac}}{2a} \\
x &= \frac{-b + \sqrt{b^2 - 4ac}}{2a}
\end{aligned}
\]

\end{document}
